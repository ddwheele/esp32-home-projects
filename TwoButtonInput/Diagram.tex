\documentclass{article}

\usepackage[siunitx]{circuitikz}

\begin{document}

\begin{circuitikz}
	[pin/.style={rectangle, draw, inner sep=0pt, minimum height=0.5cm, minimum width=1cm}]

	%\clip(0,-10) rectangle (15,10);
	%\draw[step=1cm,gray,ultra thin] (0,-10) grid (20,10);
	%\draw (0,0) circle [radius=10pt];
		
	% Body of the chip
	\node at (1,0) [rectangle, thick, draw, minimum width=3cm, minimum height=16cm]  {};
	\node at (0.5,7) [font=\bf]{ESP32};
	% Pins
	\node (VIN) at (2,7) [pin] {VIN} ;
	\node (GND) at (2,6) [pin]{GND} +(1,.5);
	\node (D26) at (2,1) [pin] {D26} +(1,.5);
	\node (D25) at (2,0) [pin] {D25} +(1,.5);
	\node (D33) at (2,-1) [pin] {D33} +(1,.5);
	\node (D32) at (2,-2) [pin] {D32} +(1,.5);
			
	\draw (GND.east) -- ++(10,0) -- ++ (0,-10) node(GROUND)[ground]{};
	
% draw the button circuits
	\draw (VIN.east)  to[short,-*] ++(1,0)
		to[normally open push button, draw, color=red] ++(0,-6)
		to[short]  +(0, -1)
		to[short]  (D25.east);  
	 \draw (3.5,0)  to[R=10<\kilo \ohm>,*-*] +(9,0); 
	  
	\draw (VIN.east)  to[short] ++(2,0)
		to[normally open push button, draw, color=green] ++(0,-6)
		to[short]  +(0, -3)
		to[short]  (D32.east);  
	 \draw (4.5,-2)  to[R=10<\kilo \ohm>,*-*] +(8,0); 

% draw the LED circuits
	\draw (D26.east)  to[leDo, color=red] ++(8,0) 
		   to [R=\SI{220}{\ohm}]   ++(1,0)
		   	  to[short, -*]  ++(1,0);
		 
	\draw (D33.east)  to[leDo, color=green] ++(8,0) 
		   to [R=\SI{220}{\ohm}]   ++(1,0)
		  to[short, -*]  ++(1,0);

\end{circuitikz}

\end{document}