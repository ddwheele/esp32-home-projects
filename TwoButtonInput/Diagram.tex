\documentclass{article}

\usepackage[siunitx]{circuitikz}

\begin{document}

\begin{circuitikz}
	\ctikzset{multipoles/thickness=3}
	\ctikzset{multipoles/dipchip/width=2}
	\clip(0,-10) rectangle (15,10);
	\draw[step=1cm,gray,very thin] (0,-10) grid (20,10);
	\draw (0,0) circle [radius=10pt];
	
	% Body of the chip
	\draw(0,-8.5) rectangle (3,8.5);
	% Pins
	\draw (2,7.5) rectangle node{VIN} +(1,.5) ;
	\draw (2,6.5) rectangle node{GND} +(1,.5) ;
	\draw (2,1) rectangle node{D26} +(1,.5) ;
	\draw (2,-.25) rectangle node{D25} +(1,.5) ;
	\draw (2,-1.5) rectangle node{D33} +(1,.5) ;
	\draw (2,-2.5) rectangle node{D32} +(1,.5) ;
	
	
			
	\draw (C.bpin 30) -- ++(1,0) node[vcc]{};
%	\draw (C.bpin 29) -- ++(10,0) -- ++ (0,-15) node[ground]{};
	
% draw the button circuits
%	\draw (C.bpin 30)  -- ++(1,0) -- ++(0,-3)
%		 to [nopb, color=red] 	++( 2,-0) 
 %		to[short]  ++(0,-5)
%		 to[short,-*]  (C.bpin 23);
%		  
%	\draw (C.bpin 30)  -- ++(1,0) -- ++(0,-5)
%		 to [nopb,color=green] ++( 2,-0) 
%		  to [R=\SI{10}{k\ohm}]   ++(6,0)
%		  to[short, -*]  ++(1,0);

	%\draw (C.bpin 23) -- ++(3,0)
	%	 -- ++(0,3)
%		  to[short, -*]  ++(0,2);

%	\draw (C.bpin 21) -- ++(3,0) to [nopb,color=green] 
%		    ++(2,0) to [R=\SI{10}{k\ohm}]
%		 ++(3,0) node[ground]{};

% draw the LED circuits
%	\draw (C.bpin 24)  to[leDo, color=red] ++(8,0) 
%		   to [R=\SI{220}{\ohm}]   ++(1,0)
%		  to[short, -*]  ++(1,0);
		 
%	\draw (C.bpin 22)  to[leDo, color=green] ++(8,0) 
%		   to [R=\SI{220}{\ohm}]   ++(1,0)
%		  to[short, -*]  ++(1,0);

\end{circuitikz}

\end{document}