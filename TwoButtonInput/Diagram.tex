\documentclass{article}

\usepackage[siunitx]{circuitikz}

\begin{document}

\begin{circuitikz}
	\ctikzset{multipoles/thickness=3}
	\ctikzset{multipoles/dipchip/width=2}
	\clip(0,-10) rectangle (15,10);
	\draw (0,0) node[dipchip, scale=2, external pins width=0.0,
		num pins=30, hide numbers](C){ESP32};
	\node [left] at (C.bpin 30) {VIN};
	\node [left] at (C.bpin 29) {GND};
	\node [left] at (C.bpin 28) {...};
	\node [left] at (C.bpin 27) {...};
	\node [left] at (C.bpin 26) {...};
	\node [left] at (C.bpin 25) {...};
	\node [left] at (C.bpin 24) {D26};
	\node [left] at (C.bpin 23) {D25};
	\node [left] at (C.bpin 22) {D33};
	\node [left] at (C.bpin 21) {D32};
	\node [left] at (C.bpin 10) {...};
		
	\draw (C.bpin 30) -- ++(1,0) node[vcc]{};
	\draw (C.bpin 29) -- ++(10,0) -- ++ (0,-15) node[ground]{};
	
% draw the button circuits
	\draw (C.bpin 30)  -- ++(1,0) -- ++(0,-3)
		 to [nopb, color=red] ++( 2,-0) 
		 to [R=\SI{10}{k\ohm}]   ++(6,0)
		 to[short, -*]  ++(1,0);
		  
	\draw (C.bpin 30)  -- ++(1,0) -- ++(0,-5)
		 to [nopb,color=green] ++( 2,-0) 
		   to [R=\SI{10}{k\ohm}]   ++(6,0)
		  to[short, -*]  ++(1,0);

	\draw (C.bpin 23) -- ++(3,0)
		 -- ++(0,3)
		  to[short, -*]  ++(0,2);

	\draw (C.bpin 21) -- ++(3,0) to [nopb,color=green] 
		    ++(2,0) to [R=\SI{10}{k\ohm}]
		 ++(3,0) node[ground]{};

% draw the LED circuits
	\draw (C.bpin 24)  to[leDo, color=red] ++(8,0) 
		   to [R=\SI{220}{\ohm}]   ++(1,0)
		  to[short, -*]  ++(1,0);
		 
	\draw (C.bpin 22)  to[leDo, color=green] ++(8,0) 
		   to [R=\SI{220}{\ohm}]   ++(1,0)
		  to[short, -*]  ++(1,0);

\end{circuitikz}


\end{document}