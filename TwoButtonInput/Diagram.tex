\documentclass{article}

\usepackage[siunitx]{circuitikz}

\begin{document}

\begin{circuitikz}
	[pin/.style={rectangle, draw, inner sep=0pt, minimum height=0.5cm, minimum width=1cm}]

	%\clip(0,-10) rectangle (15,10);
	\draw[step=1cm,gray,very thin] (0,-10) grid (20,10);
	\draw (0,0) circle [radius=10pt];
		
	% Body of the chip
	\draw(0,-8) rectangle (2.5,8);
	% Pins
	\node (VIN) at (2,7) [pin] {VIN}  ;
	\node (GND) at (2,6) [pin]{GND} +(1,.5) ;
	\node (D26) at (2,1) [pin] {D26} +(1,.5) ;
	\node (D25) at (2,0) [pin] {D25} +(1,.5) ;
	\node (D33) at (2,-1) [pin] {D33} +(1,.5) ;
	\node (D32) at (2,-2) [pin] {D32} +(1,.5) ;
			
%	\draw (C.bpin 30) -- ++(1,0) node[vcc]{};
%	\draw (C.bpin 29) -- ++(10,0) -- ++ (0,-15) node[ground]{};
	
% draw the button circuits
%	\draw (C.bpin 30)  -- ++(1,0) -- ++(0,-3)
%		 to [nopb, color=red] 	++( 2,-0) 
 %		to[short]  ++(0,-5)
%		 to[short,-*]  (C.bpin 23);
%		  
%	\draw (C.bpin 30)  -- ++(1,0) -- ++(0,-5)
%		 to [nopb,color=green] ++( 2,-0) 
%		  to [R=\SI{10}{k\ohm}]   ++(6,0)
%		  to[short, -*]  ++(1,0);

	%\draw (C.bpin 23) -- ++(3,0)
	%	 -- ++(0,3)
%		  to[short, -*]  ++(0,2);

%	\draw (C.bpin 21) -- ++(3,0) to [nopb,color=green] 
%		    ++(2,0) to [R=\SI{10}{k\ohm}]
%		 ++(3,0) node[ground]{};

% draw the LED circuits
%	\draw (C.bpin 24)  to[leDo, color=red] ++(8,0) 
%		   to [R=\SI{220}{\ohm}]   ++(1,0)
%		  to[short, -*]  ++(1,0);
		 
%	\draw (C.bpin 22)  to[leDo, color=green] ++(8,0) 
%		   to [R=\SI{220}{\ohm}]   ++(1,0)
%		  to[short, -*]  ++(1,0);

\end{circuitikz}

\end{document}